\documentclass[16pt]{article}
\usepackage[english,russian]{babel}
\usepackage[a4paper, margin=0.5cm, bottom=1.5cm]{geometry}
\usepackage{microtype}
\usepackage{ragged2e}
\usepackage{graphicx}
\usepackage{tabularx}
\usepackage{icomma}
\usepackage{lipsum} 
\usepackage{amssymb}
\usepackage{paracol}
\usepackage{amsmath}
\usepackage{fancyhdr}
\usepackage{xcolor}
\usepackage{diagbox}
\usepackage{tikz}
\usepackage{geometry}
\usepackage{indentfirst}
\pagecolor[rgb]{1, 1, 0.93}
\graphicspath{{./images}}
\columnratio{0.5,0.5}
\newcommand{\redbox}{
\tikz[baseline=-0.5ex]{
    \fill\definecolor{bone}{rgb}{1,1,0.93} (0,0) rectangle (0.8,0.3);
    \draw[red, line width=1.5pt] (0,0) rectangle (0.8,0.3);
}
}

\begin{document}
    \pagestyle{fancy}
    \fancyhead{}
    \fancyfoot{}
    \pagenumbering{arabic}
    \fancyfoot[L]{\textbf{\thepage}}
    \setcounter{page}{18}
    \begin{paracol}{2}
        \begin{column}\large            
            б) если \(n=4s+1\), то на каждой вертикали и на каждой горизонтали доски располагаются ровно \(s=\left[\frac{n}{4}\right]\) точек (королей) из \(M\) (здесь \([a]\) --- целая часть числа а);\\
            \indent в) если \(n=4s+3\), то на каждой вертикали и на каждой горизонтали торической доски располагаются либо \(s\), либо \(s+1\), точек из \(M\).\\
            \indent\textls[350]{Задача 5}. \ Будем изображать граф \(P^2_5\) (тор) квадратом, помня, что противоположные стороны его склеены. Назовем \textit{циклическим сдвигом} графа \(P^2_n\) отображение «параллельный перенос», при котором элемент \((0; 0)\) переходит в \((s; t)\).\\
            \indent Докажите, что любое н. н. м. \(M\) графа \(P^2_5\) можно привести к виду, изображенному на рисунке 5, если разрешить циклические сдвиги графа доски и симметрии относительно диагонали, вертикали и горизонтали квадрата.\\
            \indent\textls[350]{Примечание}. \ Симметрию относительно диагонали \((0; 0) \ (n-1; n-1)\) квадрата можно записать формулой \((x; y)\to(y; x)\). Цикличекий сдвиг можно было бы записать так: \((x; y) \to (x+s; y+1)\), но в этой бы отождествить числа \(n\) и \(0\), \(n+1\) и \(1\) и т. д. Для этого применяются значки \(\equiv\) и \(mod \ n\). Говорят, что \(a\) сравнимо с \(b\) по модулю \(n\) и пишут \(a\equiv b \ (mod \ n)\), если \(a\) и \(b\) при делении на \(n\) дают равные остатки. Обозначим через \(a \ (mod \ n)\) остаток от деления \(a\) на \(n\). Тогда можно сказать, что элемент \((x; y)\) переходит при циклическом сдвиге графа в элемент \(((x+s) \ (mod \ n); \ (y + 1)\) \((mod \ n))\), а при отражении относительно, скажем, прямой \(x = a\) --- в элемент \((2a-x) \ (mod \ n); y)\).
            \begin{figure}[h!]
                \centering
                \includegraphics[width=1\linewidth]{cube.png}
            \end{figure}
        \end{column}
        \begin{column}\Large
            \noindent\textbf{Все, что известно}\\
            \noindent Попробуем дальше усложнить наш передающий аппарат; построим аппарат \(A^k\), где \(k\) --- некоторое натуральное число. Это значит, что с помошью аппарата \(A\) мы будем передавать пачки из \(k\) букв, каждая из которых берется из начального входного алфавита \(S\).\\
            \indent По аналогии с двухбуквенными сигналами несложно построить граф \(G^k\) ошибок аппарата \(A^k\). Его множество вершин --- это алфавит \(S^k\), состоящий из всевозможных наборов букв длины \(k\): \((v_1; \ v_2; \ ...; \ v_k)\), где все \(v_i\) берутся из алфавита \(S\). Несложно построить и ребра в графе \(G^k\), то есть понять, какие сигналы \((v_1; \ v_2; \ ...; \ v_k)\) и \((w_1; \ w_2; \ ...; \ w_k)\) могут перепутаться. Для этого должны перепутаться между собой буквы каждой из координат, то есть для каждого \(i \ (1 \leq i \leq k)\) должно выполняться одно из двух условий: либо \(v_1 = w_1\), либо \(v_1 \sim w_1\) (если \(v_i = w_i\) для каждого \(i\), то данные наборы совпадают).\\
            \indent Определять точное назначение пропускной способности аппарата \(A^k\) в общем случае довольно сложно, но можно оценить ее снизу.
            {\large
            \indent\textls[350]{Задача 6}. Докажите, что\\
            \centerline{\(\alpha \ (G^k)\geq(\alpha(G))^k\)}}\\
            \indent А дальше... Даже в одном из самых простых случаев, когда граф ошибок --- это \(n\)-угольник \(P^n\), об \(\alpha \ (P^k_n)\) известно мало. Все, что известно из литературы, сейчас и будет рассказано.\\
            \indent Торическую шахматную доску размером \(n \times n\) называют еще \textit{двумерным тором}. Аналогично \(k\)-ю степень \(n\)-угольника (граф \(P^k_n\)) можно назвать \(k\)-мерным тором со стороной \(n\).
        \end{column}
    \end{paracol}
    \begin{figure}[h!]
        \centering
        \includegraphics[width=0.9\linewidth]{tables.png}
        \\
        Рис. 6
    \end{figure}
    \begin{paracol}{2}
        \begin{column}\Large            
            \noindent\resizebox{1\linewidth}{!}{\begin{tabular}{|r|rrrrr}
                 \hline
                 \diagbox{\(n\)}{\(k\)} & 1 & 2 & 3 & 4 & 5 \\
                 \hline
                 1 & 1 & 1 & 1 & 1 & 1 \\
                 2 & 1 & 1 & 1 & 1 & 1 \\
                 3 & 1 & 1 & 1 & 1 & 1 \\
                 4 & 2 & 4 & 8 & 16 & 32 \\
                 5 & 2 & 5 & 10 & 25 & \redbox \\
                 6 & 3 & 9 & 27 & 81 & 243 \\
                 7 & 3 & 10 & 33 & \redbox & \redbox \\
                 8 & 4 & 16 & 64 & 256 & 1024 \\
                 9 & 4 & 18 & 81 & \redbox & \redbox \\
                 10 & 5 & 25 & 125 & 625 & 3125 \\
                 11 & 5 & 27 & \redbox & \redbox & \redbox \\
                 12 & 6 & 36 & 216 & 1296 & 7776 \\
                 13 & 6 & 39 & \redbox & \redbox & \redbox \\
                 14 & 7 & 49 & 343 & 2401 & 16807 \\
                 15 & 7 & 52 & \redbox & \redbox & \redbox
            \end{tabular}}
            \\
            \\
            \noindent Табл. 1\\
            \\
            Вершины графа \(P^k_n\) можно отождествить с набором из \(k\) целых чисел: \((x_1; \ x_2; \ ...; \ x_k)\), где каждое число \(x_i\) изменяется от \(0\) до \(n - 1\).\\
            \indent{\large\textls[350]{Задача 7}. Определите числов вершин в графе \(P^k_n\).}\\
            \indent Согласно определению, два набора \((x_1; \ x_2;\ \\ \ ...; \ x_k)\) и \((y_1; \ y_2; \ ...; \ y_k)\) смежны в графе \(P^k_n\), если в \(n\)-угольнике каждая пара координат \(x_i, y_i\) --- соседи, то есть для каждого \(i\) значения \(i\)-той координаты различаются не более, чем на единицу: \(|x_i -y_i|\equiv h_i \ (mod \ n) \in \{0, 1\}\) для всех \(i\) от \(1\) до \(k\) (например, на рисунке 6 для случая \(k=3, n=5\) отмечены соседи набора \((0; \ 1; \ 2)\)).\\
            \indent{\large\textls[350]{Задача 8}. Определите число соседей каждой вершины \(k\)-мерного тора.}\\
            \indent{\large\textls[350]{Задача 9}. Докажите, что}\\
            \centerline{\(\alpha \ (P^k_n)\leq [(\frac{n}{2})^k]\)}\\
        \end{column}
        \begin{column}\Large
            \indent{\large\textls[350]{Задача 10}. Определите \(\alpha \ (P^k_n)\), если \(n\) --- четное.}\\
            \indent Для нечетных \(n\) справедлива оценка\\
            \centerline{\((\frac{n-1}{2})^k \leq \alpha \ (P^k_n) \leq [(\frac{n}{2})^k]\),}\\
            а иногда известно и точное значение \(\alpha \ (P^k_n)\).\\
            \indent\textls[350]{Теорема}. \textit{Если \(n-1\)} делится на \(2^k\), то\\
            \centerline{\(\alpha \ (P^k_n) = \frac{n-1}{2^k}\cdot n^{k-1}\).}\\
            \indent Полное доказательство этой теоремы довольно длинно, поэтому покажем лишь, как определить независимое множество с таким числом элементов. По каждому набору \((x_1; \ x_2;\ \\ \ ...; \ x_{k-1})\) первых \(k-1\) координат получим \(r=\frac{n-1}{2^k}\) значений \(k\)-той координаты:\\
            \noindent \(x_k \equiv 2l + r(2^{2k-1}x_1 +\) \\  \indent\( +2^{2k-2}x_2 + ...+2^1x_{k-1}) \ (mod \ n)\).\\
            \noindent Здесь \(l\) --- любое целое число от \(0\) до \(r - 1\).
            \indent{\large Если для случая \(k-2\) вы изобразите множество вершин, координаты которых заданы последней формулой, то получится ответ задачи M415 при \(n-1\), делящимся на 4.}\\
            \indent Эти результаты позволяют построить таблицу значений \(\alpha \ (P^k_n)\) (рис. 7).\\
            \noindent В ней часть чисел получена теоретически, а остальные --- при помощи вычислительной машины. Вы видите, что с ростом \(k\) и \(n\) доля незаполненных мест становится все весомее.\\
            \indent Может быть, кто-то из вас заполнит пропуски в этой таблице и решит задачу об определении пропускной способности передающих аппаратов?
        \end{column}
    \end{paracol}
\end{document}
